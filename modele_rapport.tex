%%%%%%%%%%%%%%%%%%%%%%%%%%%%%%%%%%%%%%%%%%%%%%%%%%%%%%%%%%%%%%%%%%%%%%%%%%%%%%%
\documentclass{article}

\usepackage[french]{babel}
\usepackage[utf8]{inputenc}
\usepackage[T1]{fontenc}

% inclure des images
\usepackage[pdftex]{graphicx}


\title{Rapport d'Algorithmique Avancée \\ Problème de l'arbre de Steiner}

\author{ Your name(s) }

\begin{document}


\maketitle

\newpage



\tableofcontents

\newpage


% N'hésitez pas à illustrer votre rapport avec des figures.


\section{Le Problème de l'arbre de Steiner}




\subsection{Description et modélisation}

Présenter le problème en termes simples.
Puis proposer une modélisation formelle.

\subsection{Complexité}

Présenter la preuve que le problème est NP complet


\section{Approximation}

\subsection{Algorithme}
Présenter l'algorithme d'approximation

\subsection{Rapport d'approximation}
Donner la preuve du rapport d'approximation


\section{Métaheuristique}

Présenter la métaheuristique que vous avez choisi d'implémenter ainsi que les choix que vous avez effectués pour la représentation d'une solution, la fonction de voisinnage et la fonction d'évaluation



\section{Résultats expérimentaux}

Dans cette section, comparez les résultats de vos algorithmes d'approximation et heuristiques à l'aide de courbe et de simulation bien choisies. Les courbes doivent être moyennées et l'intervalle de confiance doit être représenté.

Justifiez à le choix des paramètres de vos meta-heuristiques, vous vous appuierez sur des courbes. 

Pour les meta-heuristiques, les résultats doivent montrer l'évolution du score de vos algorithme au cours du temps.

Présenter les résultats que vous obtenez au minimum sur les graphe B02, B04 et idéalement sur une série de graphes. 


\section{Conclusion}

\section{Bibliographie}


\end{document}

